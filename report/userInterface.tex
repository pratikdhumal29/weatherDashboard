\section{User interface}
\subsection{How to use it}
The user interface is divided in two parts: the map and the graph. The map is a visualization of all the US stations or the following computations: temperature anomalies, snow cumulation, rainfall and storms. The graph is a visualization of the nearest neighbors.

Below the map we can select which layer we want to show between the one we mentioned. Just below we can specify the date for which we want to see the data\footnote{This HTML input is of type \texttt{date} which is a new type of input from HTML5 and only working well with Google Chrome.}. However data visualized on the map will not always be for the exact day we entered in the field. Day by day data is only available for temperature anomalies. Here's a table that summarizes which type of period is available for which layer.
\begin{center}
\begin{tabular}{|l|c|c|c|c|}\hline \textbf{Layer} & \textbf{Day} & \textbf{Week} & \textbf{Month} & \textbf{Year} \\\hline Temperature anomalies & $\times$ &  & $\times$ &  \\\hline Snow cumulation &  & $\times$ & $\times$ &  \\\hline Rainfall & $\times$ &  & $\times$ &  \\\hline Storms &  &  & $\times$ & $\times$ \\\hline \end{tabular}
\end{center}

Next to each “computed” layer label, the indication in parenthesis says what type of period is currently used. For instance, if we select “Day” below the date field, then temperature anomalies and rainfall will obviously show their daily data, but Snow cumulation will show its weekly data and storms its monthly data.

When ticking the Temperature anomalies checkbox, additional settings appear. They allow the user to switch between anomalies at average temperatures, maximum temperatures or minimum temperatures. A last option allow us to display only extreme temperature anomalies.

Finally, there's the Wikipedia articles list. This list contains only articles on US weather events and is automatically updated to match the date entered in the field. This feature allow the user to easily find a match between the data shown in the map and a Wikipedia article.

\subsection{Map engine}
At first we tried to include maps using OpenLayers 3, which is quite promising but still in beta. We faced some problems due to different projections system which gave us some hard time. Hence we finally switched to Leaflet + Mapbox which works pretty well.
