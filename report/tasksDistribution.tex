\section{Tasks distribution}
\begin{longtable}{|l|p{10.5cm}|}
\hline
\textbf{Student} & \textbf{Tasks s-he accomplished} \\
\hline
Aubry Cholleton & \begin{itemize}
	\item Find NOAA datasets on the web and resources about it.
	\item Write a generic algorithm to filter and aggregate data from NOAA (temporally and spatially), perform some statistics and classification of extreme events and output the results.
	\item Apply this algorithm to temperature anomalies.
	\item Write python scripts to convert Hadoop output to geoJSON, to transform coordinate systems and to interpolate data.
	\item Visualizations of temperature anomalies in javascript using leaflet for the map and geoJSON file format.
	\item Integrate visualization of rainfall events.
	\item Project leading and presentation.
\end{itemize}\\
\hline
Jonathan Duss & \\
\hline
Anders Asheim Hennum & \begin{itemize}
	\item Developed and implemented $k$-nearest neighbor with Map Reduce
	\item Formatted and analyzed output with R
	\item Converted output to json format and visualized data as graph with D3.js
\end{itemize}\\
\hline
Alexis Kessel & \begin{itemize}
	\item Find and extract location, date, nature of extreme weather events from Weather data. Overall, locate places which tend to get more extreme weather events.
	\item Extract Rain information from NOAA datasets and filter inconsistent rain data.
	\item Write a generic algorithm to aggregate temporally extreme events.
	\item Find matching extreme events on map, both temporally and spatially, with the Wikipedia articles.
\end{itemize}\\
\hline
Quentin Mazars-Simon & \begin{itemize}
	\item Data retrieval from Wikipedia and upload to our servers
	\item Wikipedia articles filtering based on categories
	\item Wikipedia information extraction
	\item Wikipedia visualisation (Python + Javascript)
	\item Project presentation
\end{itemize}\\
\hline
Cédric Rolland &  \begin{itemize}
	\item Find and extract station id, location, date, wind speed and nature of extreme weather events from NOAA datasets.
	\item Write a generic algorithm to find extreme events period and duration, start date and end date.
	\item Write a generic algorithm to cluster stations within a same extreme event.
	\item Retrieve final information of all extreme events (start/end date, location, radius, etc.) in order to to give them to a post-processing algorithm.
	\item Find matching extreme events on map, both temporally and spatially, with the Wikipedia articles.
	\item Write python scripts to order data by month and by year for every extreme events.
\end{itemize}\\
\hline
Orianne Rollier & \\
\hline
David Sandoz &
\begin{itemize}
	\item Data retrieval from the Integrated Surface Database to our server for 39 years of USA data and 10 years of Switzerland data.
	\item User interface
	\item Post-processing and visualization of the storms
	\item Minutes writing of every scheduled meeting of the team
	\item Layout of the final report
	\item Slides of the presentation
\end{itemize}\\
\hline
Amato Van Geyt & \begin{itemize}
	\item Rainfall Anomaly Detection (Java).
	\item Presenting the Presentation
	\item Post-processing to JSon (Python)
	\item Correcting and finalising report
\end{itemize}\\ 
\hline
\end{longtable}
