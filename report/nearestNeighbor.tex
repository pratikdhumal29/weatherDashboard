\section{Nearest Neighbors}

\subsection{Why?} % (fold) \label{sub:why_}

% subsection why_ (end)
The main purpose of implementing nearest neighbor, is to find similar weather
periods in the history and possibly find patterns in the climate. This is useful
in terms of understanding the climate and to forecast specific events. 

As an example, here in Switzerland, it could be useful in terms of avalanche
danger. If most of the nearest neighbors of the current winter so far, led to
many and massive avalanches, it is reason to believe that also the current
winter can become a such a winter. By knowing this, it would be possible to take
action before the events occur. The same method can be used to forecast events
like floods or drought.  

It can also be used to find patterns in the climate such as cycles and periodic
structures. 

TODO: Rewrite to better language

\subsection{Implementation} % (fold) \label{sub:implementation}

The k-nearest neighbor is a fairly simple algorithm. You choose a metric,
calculate the distance from the reference node to all the other's, and pick the
$k$ nearest neighbors. This has been done more or less straight forward.

In this case a node is a period of time. We fixed the period to one month, such
that you can pick one month for a specific year, and then find the most similar
months for the other years. Often climate data is summarized by month, so this
makes it easy to find data to compare with.

Next question is to determine how many intervals the period of time should be
divided into. Should we compare averages for an hour, a day, a week, or just one
value for the whole month? For a small region (a few stations) and a short
period of time (a day), then comparing hour by hour could make sense and give
good results. But for a large region (thousands of stations) and for a longer
period (a month), this will mostly just return noise. Here, the method is
implemented with a period option so one can choose how many periods a month
should be divided into. The final results is run with this set to 1, meaning
that averages for the whole month is used. 

TODO: add figure shows how this was implemented with map reduce on Hadoop. Write
some more and clean language.
% subsection implementation (end)

\subsection{Results} % (fold) \label{sub:results}

TODO: Present some results with some figures to show that it works.

% subsection results (end)
